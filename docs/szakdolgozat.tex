\documentclass[twoside, a4paper, 12pt]{article}
\usepackage{thesis-style}

% Töltsd ki a saját szakdolgozatod adataival
\def\CIM{Cache optimalizált lineáris szita párhuzamos megvalósítása}
\def\SZERZO{Huszti Péter}
\def\VEDESEVE{2017}

\def\TANSZEK{Komputeralgebra Tanszék}
\def\TEMAVEZETO{Dr Vatai Emil}
\def\TEMAVEZETOBEOSZTAS{adjunktus}


\title{\CIM}
\author{\SZERZO}
\date{\VEDESEVE}

\begin{document}
\pagestyle{empty}

% belső fedőlap
\include{tex/fedolap}
\cleardoublepage

% a belső fedőlap utáni lap a témabejelentő

% tartalomjegyzék
\tableofcontents
\cleardoublepage

\pagestyle{plain}
\setcounter{page}{1}

% tartalom
\input{tex/tartalom}

\section{Bevezetés}

\subsection{A dolgozat célja, motiváció}

Az alkalmazott matematikában nagyon fontos szerepet játszanak az prím számok, elég csak például a nyílt kulcsos titkosítási módszerekre gondolnunk. Ebből kifolyólag az évek során nagyon sok módszert fejlesztettek ki prímsázmok keresésére, például a Fermat-teszt, vagy a Miller-Rabin teszt. Ezen módszerekkel elég gyorsan el lehet dönteni egy darab számról, hogy prím-e vagy sem, és így ezért az ilyen algoritmusokkal nagyon nagy prímeket is meg lehet találni viszonylag gyorsan. \par
Előfordulhat azonban probléma, hogy egy adott intervallumban szeretnénk megtalálni az ott lévő összes prím számot. Az ilyen feladatok megoldására a leghatékonyabb módszerek a szitáló módszerek. Ezek úgy működnek, hogy kiválasztunk bizonyos számokat, és azokkal "végigszitálunk" a vizsgált intervallumon, és az érintetlenül hagyott számok lesznek a nekünk megfelelő, jelen esetben prím számok. Ezek közül is a legegyszerűbb egészen az ókorig nyúlik vissza, az ún. Erathosztenészi-szita. Ez az algoritmus ahhoz képest, hogy milyen rég óta ismert, meglepően hatékonyan működik. De természetesen vannak hátrányai, például hiába tűnik nagyon gyorsnak komplexitás szempontjából, elég nagy intervallumokra nagyon meg fog növekedni a memória olvasások száma, ami mint köztudott nagyságrendekkel lassabb, mint a processzorok utasítás végrehajtó képessége már a memóriában lévő adatokon. Nem meglepő módon sok féle képpen fel lehet javítani az algoritmus teljesítményét mai modern eszközökkel. \par
A dolgozat célja az, hogy a fent említett "naív" szitáló algoritmus teljesítményét növeljük, miközben a program skálázható is maradjon, és így a valós gyakorlati életben is alkalmazni lehessen. A dolgozat során két féle módon próbáltam meg javítani a teljesítményen.\par
Az első és talán legkézenfekvőbb módszer a program párhuzamosítása. Ezt viszonylag egyszerűen meg lehet tenni, mivel az algoritmust könnyedén fel lehet darabolni kisebb, egymással ekvivalens részfeladatokra, amiket szét lehet osztani a processzor szálai közt. Ráadásul a konkurens programokra jellenző veszélyek nem állnak fent, így sok nehézségtől meg tudjuk kímélni magunkat, ami a teljesítményt is javítja. \par
A másik megközelítés, hogy a memóriaműveletek számát próbáljuk meg minimalizálni. Erre egy hatékony megvalósítása az ún. COLS - cache optimalizált lineáris szita - algoritmus, aminek az megvalósítása is része a dolgozatnak.\par
Végül a fentiekből magától értetődik egy újabb gyorsítási lehetőség, hogy a COLS algoritmust is meg lehet valósítani párhuzamos szálakon, ami mint majd később látjuk megint csak nagy teljesítménynövekedéssel járhat.\par
A dolgozat során implementáltam a fent említett négy algoritmust, és azt vizsgáltam, hogy milyen esetekben (a probléma mérete, hardver specifikációja) mennyire tudják, ha egyáltalán lehetséges, felgyorsítani a fenit módszerek a probléma megoldását. Az algoritmusokat C++ nyelvben valósítottam meg, csak és kizárólag a standard C++11 szabvány által kínált lehetőségeket felhasználva.\par
Az elészített programmal könnyen és gyorsan lehet egy megadott intervallumon megkeresni a prím számokat, továbbá megfelelően paraméterezhető a rendelkezésre álló processzor(ok) tulajdonságainak ismeretében. A program jól skálázható, így nagy problémák megoldására is lehetőséget nyújt.

\clearpage
\section{Matematikai háttér}

\subsection{Prím számok, faktorizáció}

\begin{definition} Egy p természetes számot prímnek nevezünk, ha \(\forall a,b\)-re amire \(p|a \cdot b \rightarrow (p|a \vee p|b)\). \end{definition}
Természetes számok körében ez a definíció ekvivalens azzal, hogy egy prím számnak kettő, és csak kettő osztója van, 1 és önmaga. \par
A prím számok kitüntetett szerepet játszanak a matematikában. Többek közt felhasználják őket hasítótáblákhoz, pszeudovéletlen számok generálásához vagy nyílt kulcsú titkosításokhoz. Utóbbiak széles körben elterjedtek, valószínűleg sokan ismerik például az RSA kódolást, az SSH-t vagy a HTTPS-t. Ezek mind fontos részét képezik a modern kornak. A nyílt kulcsú kódolások olyan matematikai problémákon alapulnak, amelyeket megoldani nehéz, vagyis a mai eszközeinkkel valós időben nem lehetséges, viszont ellenőrizni egy lehetséges megoldást gyors és egyszerű. A leggyakrabban használt ilyen probléma a prím faktorizáció. \par
Számelmélet alaptétele: minden pozitív szám felírható egyértelműen prímszámok szorzatára. \par
Viszont, ennek a felbontásnak a megkeresése NP-nehéz probléma, vagyis nem tudunk jelenleg sokkal jobb módszert annál, mint hogy kipróbáljuk az összes lehetséges prím számot, hogy oszthója-e a felbontani kívánt számnak.\par
Tehát jól látszik, hogy a prím számok megtalálása  kiemelten fontos feladat. Rengeteg módszer létezik arra, hogy prímeket keressünk. A dolgozat az ún. szitáló módszerekkel foglalkozik, konkrétabban ezek felhasználása prímszámok keresésére. Ezeknek a módszereknek megvan az az előnye, hogy egy adott intervallumban megtalálják az összes ott előforduló prímet, viszont ha konkrétan csak egy darab számról akarjuk eldönteni, hogy prím-e, akkor ezeknél a módszereknél léteznek sokkal hatékonyabbak is.

\subsection{Eratoszthenészi-szita}

Az Eratoszthenészi-szita, mint a nevéből is látszik már egy nagyon régen ismert algoritmus. Ennek ellenére, bármilyen meglepő is, ha gyorsan meg akarjuk keresni egy intervallumban az összes prím számot, akkor ehhez az algoritmushoz kell visszanyúlnunk. Ez egy egyszerű kizárásos algoritmus. A számelmélet alaptétele szerint az intervallumunkban minden szám, amelyik nem prím, osztható nálánál kisebb prím szám(ok)kal. Tehát, ha lenne egy listánk a kisebb prímekről, akkor azoknak meg tudnánk találni az intervallumunkban lévő többszöröseit, és amely szám egyik kis prímnek sem többszöröse, az prím szám. Ez az alap ötlet. \par
Az algoritumus: \par
\begin{enumerate}
\item Készítünk egylistát a kisebb prímekről, amelyekkel majd ki fogjuk szitálni a vizsgált intervallumot. De mit is jelent az, hogy kis prímek? Jelöljük az intervallumunkat, ahol keressük a prímeket \( [a, b] \)-vel. A első ötlet természetesen, hogy vizsgálunk minden 1-nél nagyobb de \(a\)-nál kisebb prímszámot. De kicsit jobban belegondolva erre egyáltalán nincs szükség. 
\item 
\end{enumerate}


\subsection{COLS}

\clearpage
\section{Felhasználói dokumentáció}

\subsection{A program használata} \label{usage}

\subsection{A program kimenete}

\subsection{Rendszerkövetelmények}

A program 64-bites Microsoft Windows operációs rendszer használatával íródott és lett tesztelve. (Windows 7 és 10) Így az ilyen operációs rendszereken biztosan helyesen működik. \par
Ahhoz, hogy UNIX alapú operációs rendszeren, illetve 32-bites rendszereken használni tudjuk, ahhoz újra kell fordítani a programot. Lásd: \ref{build} Ezeken a rendszereken nem lett tesztelve a program, de nem használja ki a használt operációs rendszer által nyújtott lehetőségeket, ezért portolható. De természetesen ajánlott a tesztelő program használata is más rendszereken való használat előtt.

\subsection{CD tartalma, telepítés}

A \textit{docs} mappában található a programhoz tartozó dokumentáció .pdf formátumban. \par
A \textit{source} mappában található a futtatható fájl, ennek neve sieve.exe. Ezt a fájlt futtatva indíthatjuk el a programot. Ebben a mappában találhatóak továbba a program forrásfájljai is.  A program konfigurálásához lásd: \ref{usage}. A program módosításához, illetve újrafordításához lásd: \ref{build} \par
A program teszteléséhez használt fájlok a \texit{test} mappában találhatóak. Bővebben lásd: \ref{test} \par
A \textit{result} mappába helyezi el a program a kiszámított eredményeket. Természetesen a CD-re nem tud írni a program. \par
A \textit{benchmark} mappában találhatóak a programban elkészített algoritmusok segítségéhez használt fájlok. Itt található még korábbi mérési eredmények benchmark_result_dátum.xlsx néven. \par \par

Telepítés: \par
A program használható közvetlenül a CD-ről is, bár ekkor nem tudjuk a kapott ererdményeket megteinteni. Ehhez a CD tartalmát másoljuk fel a használni kívánt számítógépre, és ezek után a program a fentebb említett \textit{result} mappába fogja másolni az eredményeit.

\clearpage
\section{Fejlesztői dokumentáció}

\subsection{Felhasznált technológiák}
A program C++ programozási nyelv standard C++11-es verziójának használatával íródott. Semmilyen külső könyvtárat nem használ, így könnyen újrafordíthatja és portolhatja más rendszerre bárki. \par
A párhuzamos szálakon történő futtatáshoz a standard C++11 által kínált \textbf{thread} könyvtár van felhasználva. \par
A program fejlesztése során a verziókövetésre a GitHub online verziókövető rendszer volt használva. Egy program implementációja során kiemelten fontos egy hasonló szolgáltatás használata. Ennek segítségével könnyedén nyomon lehet követni a program fejlesztésének történetét, és egyszerűen lehet több számítógépről is folytatni a fejlesztést. A program megtalálható az alábbi helyen: https://github.com/peterhuszti/Thesis-MSc \par \par \par

A program tesztelése szintén standard C++ segítsgével lett megoldva. \par
A benchmark elkészítéséhez a Benchpress nevű frameworköt használtam. Ez egy nagyon egyszerűen és könnyen használható eszköz, amivel C++11 nyelven írt programok, vagy akár csak külön függvények és eljárások futási sebességét lehet mérni. REF
TODO lamda \par

\subsection{Fordítás} \label{build}

Ahhoz, hogy le tudjuk újra fordítani a programot, ahhoz egy C++11 kompatibilis fordítóra van szükség. A program eredetileg a g++ 5.2.0-ás verziójával lett lefordítva, de semmi akadálya annak, hogy más megfelelő fordítóprogramot használjunk. \par
Ha viszont g++-t használunk, akkor a következő javasolt g++ [-o sieve] -O3 -std=c++11 main.cpp. Az -o kapcsoló nem szükséges, ezzel csupán az elkészített futtatható fájl nevét tudjuk megadni. Az -std=c++11 paraméterrel tudjuk megadni, hogy a fordító a C++11-es szabvány szerint próbálja meg lefordítani a programot. Az -O3 paraméter egy optimalizációs paraméter. Ezekről bővebben lásd: \ref{optpar}

\subsection{A program felépítése}

\subsection{Adatszerkezetek}

\subsubsection{Szitatábla}
\subsubsection{Szegmensek}
\subsubsection{Körök és edények}

\subsection{Párhuzamos megvalósítás}

\subsection{Skálázhatóság}

\subsection{Tesztelés} \label{test}

\subsection{Továbbfejlesztési lehetőségek}

\clearpage
\section{Az algoritmusok összehasonlítása}

\subsection{Használt hardverek}

\subsection{Fordítási paraméterek} \label{optpar}

\subsection{Az eredmények}

\subsection{Konklúzió}

\clearpage
\section{Összegzés}

\end{document}







































